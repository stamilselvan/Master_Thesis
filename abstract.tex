%\addcontentsline{toc}{chapter}{Resume}
\chapter*{Abstract}
%\thispagestyle{empty}
\label{chap:abstract}

%
Technology transforms everything simpler, faster and smarter. In an attempt to evaluate future Radar processor concepts, an analysis is carried out based on a mock-up of a Radar processing algorithm. An ARM platform is chosen keeping size, weight and power in mind. Several ARM Cortex A9 processors are involved to execute the Radar processing algorithm. An analysis done by Airbus DS investigates various means of scheduling Radar application to the ARM processors. Being safety critical software, determinism in terms of worst case execution time is a crucial factor. However, the analysis concluded that the worst case execution time in ARM processors are not in the acceptable range.\\

This thesis studies the bottlenecks imposed by the application, data dependencies between the processing chains and possible improvements. The implementation focuses on the parallelism in the Radar application and schedules them in optimal way to achieve best results. As a part of the investigation, peak processor utilization, peak memory utilization, peak bandwidth utilization, worst case execution time, bottlenecks and future scope are discussed in detail. An analysis is presented based on the new implementation schemes.\\

\noindent
\textsl{\textbf{Keywords:} ARM Cortex A9, Latency, Multicore, Parallelism, Threading.} \\

\noindent
\textsl{\textbf{Prerequisite:} Basics of multicore processing.}\\
