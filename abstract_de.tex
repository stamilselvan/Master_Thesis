%\addcontentsline{toc}{chapter}{Resume}
\chapter*{Zusammenfassung}
%\thispagestyle{empty}
\label{chap:abstract_de}

Technologie macht alles einfacher, schneller und intelligenter. In einem Versuch, zukünftige Radar-Prozessor Konzepte zu bewerten, wird eine Analyse auf der Grundlage eines Mock-ups eines Radarverarbeitungsalgorithmus durchgeführt. Eine ARM-Plattform wurde aufgrund von Größe, Gewicht und Verbrauch gewählt. Mehrere ARM Cortex A9-Prozessoren sind beteiligt, um die Radarverarbeitungsalgorithmen auszuführen. Eine Analyse von Airbus DS untersucht verschiedene Arten von Ablaufplänen bei der Radar-Anwendung auf den ARM-Prozessoren. Als sicherheitskritischer Software ist Determinismus in Bezug auf Worst-Case-Ausführungszeiten ein entscheidender Faktor. Allerdings zeigen die Ergebnisse der Analyse, dass die Worst-Case-Ausführungszeit in den ARM-Prozessoren nicht im akzeptablen Bereich liegen. \\

Diese Dissertation untersucht die Engpässe, die durch die Anwendung auferlegt werden, Datenabhängigkeiten zwischen den Verarbeitungsketten und mögliche Verbesserungen. Die Umsetzung konzentriert sich auf die Parallelität in der Radar-Anwendung und einen Ablaufplan, der für  optimale Ergebnisse sorgt. Als Teil der Untersuchung werden spitzen Prozessor-Auslastung, Speicherauslastung Peak, Spitzenbandbreitennutzung, Worst-Case-Ausführungszeit, Engpässe und künftige Umfänge ausführlich diskutiert. Eine Analyse, basi-erend auf den neuen Implementierungsschemata, wird vorgstellt. \\


\noindent
\textsl{\textbf{Stichwort:} ARM Cortex A9, Latenz, Multicore, Parallelität, Threading.} \\

\noindent
\textsl{\textbf{Voraussetzung:} Grundlagen der Multicore-Verarbeitung.}\\

